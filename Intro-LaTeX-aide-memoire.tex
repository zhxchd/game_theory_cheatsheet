%%%%%%%%%%%%%%%%%%%%%%%%%%%%%%%%%%%%%%%%%%%%%%%%%%%%%%%%%%%%%%%%%%%%%%
% Source: Dave Richeson (divisbyzero.com), Dickinson College
% Version francaise par Vincent Pantaloni, prof.pantaloni.free.fr
% Traduction, correction et adaptation à la typographie française.
% 
% Une anti-seche en deux pages pour une intro rapide ou un aide mémoire des différentes fonctions. A imprimer en recto verso par exemple.
%
% Feel free to distribute this example, but please keep the referral
% to divisbyzero.com
% 
%%%%%%%%%%%%%%%%%%%%%%%%%%%%%%%%%%%%%%%%%%%%%%%%%%%%%%%%%%%%%%%%%%%%%%
%
%%%%%%%%%%%%%%%%%%%%%%%%%%%%%%%%%%%%%%%%%%%%%%%%%%%%%%%%%%%%%%%%%%%%%%

\documentclass[a4paper,10pt,landscape]{article}
\usepackage{fontspec}
\usepackage[T1]{fontenc}
\usepackage[frenchb]{babel} 
\usepackage{amssymb,amsmath,amsthm,amsfonts}
\usepackage{multicol,multirow}
\usepackage{calc}
\usepackage{ifthen}
\usepackage[landscape]{geometry}
\usepackage[colorlinks=true,citecolor=blue,linkcolor=blue]{hyperref}


\ifthenelse{\lengthtest { \paperwidth = 11in}}
    { \geometry{top=.5in,left=.5in,right=.5in,bottom=.5in} }
	{\ifthenelse{ \lengthtest{ \paperwidth = 297mm}}
		{\geometry{top=1cm,left=1cm,right=1cm,bottom=1cm} }
		{\geometry{top=1cm,left=1cm,right=1cm,bottom=1cm} }
	}
\pagestyle{empty}
\makeatletter
\renewcommand{\section}{\@startsection{section}{1}{0mm}%
                                {-1ex plus -.5ex minus -.2ex}%
                                {0.5ex plus .2ex}%x
                                {\normalfont\large\bfseries}}
\renewcommand{\subsection}{\@startsection{subsection}{2}{0mm}%
                                {-1explus -.5ex minus -.2ex}%
                                {0.5ex plus .2ex}%
                                {\normalfont\normalsize\bfseries}}
\renewcommand{\subsubsection}{\@startsection{subsubsection}{3}{0mm}%
                                {-1ex plus -.5ex minus -.2ex}%
                                {1ex plus .2ex}%
                                {\normalfont\small\bfseries}}
\makeatother
\setcounter{secnumdepth}{0}
\setlength{\parindent}{0pt}
\setlength{\parskip}{0pt plus 0.5ex}
% -----------------------------------------------------------------------

\begin{document}

\raggedright
\footnotesize

\begin{center}
     \Large{\textbf{MA4264 Cheatsheet}} \\
\end{center}
\begin{multicols*}{3}
\setlength{\premulticols}{1pt}
\setlength{\postmulticols}{1pt}
\setlength{\multicolsep}{1pt}
\setlength{\columnsep}{2pt}
\section{Static games of complete information}
\textbf{Cournot Model of Duopoly.} Companies choose quantities $q_1,q_2$ and sell at market price $P(q_1,q_2)=(a-q_1-q_2)^+$ for $a>c$. Reward is $\pi_i(q_1,q_2)=P(q_1,q_2)q_i-cq_i$. Then, best response is $R_i(q_j)=(a-q_j-c)/2$ if $q_j\leq a-c$ else $0$. The unique PSNE is $q_1^*=q_2^*=(a-c)/3$.

\textbf{Bertrand Model of Duopoly.} Two companies choose prices $p_1,p_2$ and quantity is given by $q_i(p_i,p_j)=a-p_i+bp_j$ for some $b>0$. Profit for $i$ is $\pi_i(p_i,p_j)=(a-p_i+bp_j)(p_i-c)$. 

\section{Dynamic games of complete information}

\section{Static games of incomplete information}

\section{Dynamic games of incomplete information}

\end{multicols*}

\end{document}

\href{http://amath.colorado.edu/documentation/LaTeX/Symbols.pdf}{\LaTeX\ Mathematical Symbols}\\
\href{ftp://tug.ctan.org/pub/tex-archive/info/symbols/comprehensive/symbols-letter.pdf}{The Comprehensive \LaTeX\ Symbol List}\\ 
\href{http://mirrors.med.harvard.edu/ctan/info/lshort/english/lshort.pdf}{The Not So Short Introduction to \LaTeX\ 2$\varepsilon$}\\
\href{http://www.tug.org/}{TUG: The \TeX\ Users Group}\\
\href{http://www.ctan.org/}{CTAN: The Comprehensive \TeX\ Archive Network}\\
\LaTeX\ for the Mac: \href{http://www.tug.org/mactex/}{Mac\TeX}\\
\LaTeX\ for the PC: \href{http://www.texniccenter.org/}{{\TeX}nicCenter} and \href{http://miktex.org/}{MiK\TeX}\\
\LaTeX\ online: \href{http://www.writelatex.com/}{WriteLaTeX}.